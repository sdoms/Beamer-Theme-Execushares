%!TEX program = xelatex
\documentclass{beamer}

\usepackage{blindtext}
\usepackage{smartdiagram}
\usetheme{Execushares}

\title{A Microbe Associated with Sleep Revealed by a Novel Systems Genetic Analysis of the Microbiome in Collaborative Cross Mice}
\subtitle{Journal club}
\author{Jason A. Bubier et al.}


\date{October 1, 2020}

\setcounter{showSlideNumbers}{1}

\begin{document}
	\setcounter{showProgressBar}{0}
	\setcounter{showSlideNumbers}{0}

	\frame{\titlepage}

	\begin{frame}
		\frametitle{Overview}
		\begin{center}
		\smartdiagram[bubble diagram]{systems \\genetic\\ approach,host\\ genotype, disease-related \\phenotypes, gut microbiome \\composition, gut gene \\expression}
		\end{center}
		
	\end{frame}

	\setcounter{framenumber}{0}
	\setcounter{showProgressBar}{1}
	\setcounter{showSlideNumbers}{1}
	\section{Results}
		\begin{frame}[b]
			\frametitle{Microbial community composition of incepient CC mice}
			\begin{figure}
			\centering
			\includegraphics[scale=0.3]{SupplementalFig1.pdf}
			\caption{Cecal microbial profile across mouse samples}
			\end{figure}
		\end{frame}

		\begin{frame}
			\frametitle{Microbial abundance QTL}
			\begin{itemize}
				\item 18 significant microbial abundance QTL
				\item 1.5 LOD C.I.: 2-24Mb, average of 7.5 Mb
			\end{itemize}
		\end{frame}
		
		\begin{frame}
			\frametitle{eQTL in the CC cecum}
			\begin{itemize}
				\item Used microarrays for estimates of transcript abundances
				\item QTL analysis to identify host genomic regions harboring allelic variants that influence the abundance of each probe
				\item 1641 significant QTL for probes, corresponding to 1513 genes
				\item 950 loci were \emph{cis}-eQTL, which contain polymorphisms that are proximal to transcript-coding regions \\ $\rightarrow$ useful for identifying expression regulatory mechanisms 
			\end{itemize}
		\end{frame}
		{

		\begin{frame}
			\frametitle{Genetic correlation of microbial abundance to disease-related traits reveals a microbe associated with sleep}
			\begin{itemize}
				\item 122 disease-related behavioral and physiological phenotypes tested
				\item 45 trait-microbe correlations, 26 exceeded the multiple testing FDR threshold
				\item 41 contained sleep phenotypes with 10 different microbes, 22 (\emph{q} < 0.05)
				\item OTU 273 \textit{Odoribacter} correlated with 21 phenotypes comparison-wise and 13 family-wise adjusted
			\end{itemize}
		\end{frame}
		
		
		\begin{frame}
			\frametitle{Genetic regulation of the abundance of \textit{Odoribacter}}
			\begin{itemize}
			\item \textit{Micab7}: NZO allele associated with increased abundance \textit{Odoribacter}, is obese and prone to diabetes
			\item Previous studies: diabetic \textit{db/db} mice and abnormal sleep pattern
				\begin{block}{Hypothesis}
				\textit{Odoribacter}, \textit{Lepr} and sleep are connected through a common mechanism. 

				\end{block}
				We expect an overlap between one or more of the QTL positional candidates and the \textit{Lepr} pathway, and that perturbations of the gut microbiota of \textit{db/db} mice should affect sleep patterns.
			\end{itemize}
			
		\end{frame}
		
		\begin{frame}
		\frametitle{Genetic regulation of the abundance of \textit{Odoribacter}}
		\begin{center}
		\smartdiagramset{descriptive items y sep=90pt}
			\smartdiagram[descriptive diagram]{
				{Ingenuity Pathway Analysis, 42 positional candidates with Lepr: \textit{Nr2f2} and \textit{Igf1r} as most likely},
{Causal graphical models G-P, {Infer the direct and the indirect associations among results of the IPA, the leptin pathway, and sleep}},
}


\end{center}



		\end{frame}
\begin{frame}
	\frametitle{Genetic regulation of the abundance of \textit{Odoribacter}}
	\begin{figure}
			\includegraphics[scale=1.3]{2E.png}
			\caption{Inferred network}
	\end{figure}

\end{frame}

\begin{frame}[b]
\frametitle{Broad-spectrum antibiotic treatment alters sleep patterns in Lepr\textsuperscript{db}Lepr\textsuperscript{db} mice}

 \begin{figure}
 \centering
 \includegraphics[scale=0.3]{FigureS4.pdf}
 \caption{Mice treated with antibiotics from conception. 
 This showed a genotype-specific effect on sleep architecture.}
 \end{figure}
\end{frame}

\begin{frame}[b]
\frametitle{Broad-spectrum antibiotic treatment alters sleep patterns in Lepr\textsuperscript{db}Lepr\textsuperscript{db} mice}

 \begin{figure}
 \centering
 \includegraphics[scale=0.4]{F3.large.jpg}
 \caption{Mean and SE for the percentage sleep time over a 5-day test, with cyclic patterns characterized on the right by an FFT of the mean sleep percentage time series.}
 \end{figure}
\end{frame}



	\section{Conclusion}
		\begin{frame}
			\frametitle{Conclusion}
			\centering
			\begin{itemize}
			\item They show for the first time a relationship between abundance of a specific microbe and sleep
			\begin{itemize}
			\item OTU273 \textit{Odoribacter} abundance is associated with \textit{Micab7} QTL and was correlated with multiple sleep phenotype measures
			\item Genomic network analyses revealed that the primary candidate gene for the QTL is \textit{Igf1r}
			\item Perturbation of this pathway in the db/db Lepr mutant mouse is associated with abnormal phenotype and an elevated abundance of \textit{Odoribacter} (among other microbes)
			\item Both of these phenomena can be restored by antibiotic treatment 
			\end{itemize}
			\end{itemize}
		\end{frame}

		\begin{frame}
			\frametitle{Outlook}
			\begin{itemize}
			\item They have not yet demonstrated that \textit{Igf1r} variation is i=the specific causal regulator 
			\item whether the locus is associated with abnormal Odoribacter abundance
			\item whether inoculation of \textit{Lepr} or \textit{Igf1r} mice with \textit{Odoribacter} and its metabolites influences sleep
			\end{itemize}
		\end{frame}


	\appendix
	
	\backupbegin
	  \begin{frame}[b]
	    \frametitle{SV35}
	    \begin{figure}
	    \includegraphics[scale=0.5]{SV35-P-chr7_bin1_region_plot.pdf}
	    \end{figure}
	  \end{frame}
	  
	  	  \begin{frame}[b]
	    \frametitle{SV33}
	    \begin{figure}
	    \includegraphics[scale=0.5]{SV33-add.P-chr7_bin1_region_plot.pdf}
	    \end{figure}
	  \end{frame}
	  
	  	  	  \begin{frame}[b]
	    \frametitle{SV33}
	    \begin{figure}
	    \includegraphics[scale=0.4]{SV33_marker_UNC13083565_RNA.pdf}
	    \end{figure}
	  \end{frame}
	  
	  	  \begin{frame}[b]
	    \frametitle{Genus Paraprevotella}
	    \begin{figure}
	    \includegraphics[scale=0.4]{G_Paraprevotella_marker_UNC13034282_RNA.pdf}
	    \end{figure}
	  \end{frame}


	\backupend

\end{document}
